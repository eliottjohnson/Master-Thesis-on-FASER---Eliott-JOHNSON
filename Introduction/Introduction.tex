%!TEX root = ../thesis.tex
%*******************************************************************************
%*********************************** First Chapter *****************************
%*******************************************************************************

\chapter{High energy physics and the LHC machine}  %Title of the First Chapter

\ifpdf
    \graphicspath{{Introduction/Figs/Raster/}{Introduction/Figs/PDF/}{Introduction/Figs/}}
\else
    \graphicspath{{Introduction/Figs/Vector/}{Introduction/Figs/}}
\fi


%********************************** %First Section  **************************************
\section{Introduction - What we are looking for - Axions, DM, BSM} %Section - 1.1 

Lorem Ipsum is simply dummy text of the printing and typesetting industry (see 
Section~\ref{section1.3}). Lorem Ipsum has been the industry's 
standard dummy text ever since the 1500s, when an unknown printer took a galley 
of type and scrambled it to make a type specimen book. It has survived not only 
five centuries, but also the leap into electronic typesetting, remaining 
essentially unchanged. It was popularised in the 1960s with the release of 
Letraset sheets containing Lorem Ipsum passages, and more recently with desktop 
publishing software like Aldus PageMaker including versions of Lorem 
Ipsum.

The most famous equation in the world: $E^2 = (m_0c^2)^2 + (pc)^2$, which is 
known as the \textbf{energy-mass-momentum} relation as an in-line equation.

A {\em \LaTeX{} class file}\index{\LaTeX{} class file@LaTeX class file} is a file, which holds style information for a particular \LaTeX{}.


\begin{align}
CIF: \hspace*{5mm}F_0^j(a) = \frac{1}{2\pi \iota} \oint_{\gamma} \frac{F_0^j(z)}{z - a} dz
\end{align}


\nomenclature[a-F]{$F$}{complex function}                                                   % first letter A is for Roman symbols
\nomenclature[g-p]{$\pi$}{ $\simeq 3.14\ldots$}                                             % first letter G is for Greek Symbols
\nomenclature[g-i]{$\iota$}{unit imaginary number $\sqrt{-1}$}                      % first letter G is for Greek Symbols
\nomenclature[g-g]{$\gamma$}{a simply closed curve on a complex plane}  % first letter G is for Greek Symbols
\nomenclature[x-i]{$\oint_\gamma$}{integration around a curve $\gamma$} % first letter X is for Other Symbols
\nomenclature[r-j]{$j$}{superscript index}                                                       % first letter R is for superscripts
\nomenclature[s-0]{$0$}{subscript index}                                                        % first letter S is for subscripts


%********************************** %Second Section  *************************************

\section{LHC machine} %Section - 1.2





%********************************** % Third Section  *************************************
\section{How does FASER fit in the context of ATLAS - low Pt - forward region}  %Section - 1.3 
\label{section1.3}

FASER (ForwArd Search ExpeRiment at the LHC) is a proposed experiment dedicated to search for light and extremely weakly interacting particles. It will be situated 480[m] along the line-of-sight of the proton collisions in front of the ATLAS interaction point at the LHC. It has the prospects of discovering new particles such as dark photons, dark Higgs bosons, heavy neutral leptons and axion-like particles. FASER is part of the CERN's Physics Beyond Colliders study group which is a exploratory study aimed at exploiting the full scientific potential of CERN's accelerator complex and it's scientific infrastructure through projects complementary to the LHC, HL-LHC and other possible future colliders.

The four main LHC detectors that focus on TeV-scale particles with high transverse momentum ($p_{T}$) may be misguided in the search for new particles. An area that they don't cover is the search for light particles with masses in the MeV to GeV range with low $p_{T}$. These extremely weakly-coupled new particles may travel hundreds of meters without interacting with any material before decaying into known SM particles. They would not be bent by magnets. Instead, they will continue along a straight line and their decay products can be spotted by the FASER experiment. These hypothetical new particles would be long lived (LLP) and very collimated with the beam (of the order of the milliradian) allowing a small and inexpensive detector.
\[
\theta\sim\Lambda_{QCD}\sim\frac{m_{B-mesons}}{E}
\]
This implies that around 500 meters after the interaction point the downstream particle spread is 10 to 100 centimeters.

\subsection{New physics}

The FASER collaboration hopes to discover new hypothetical particles through this experiment. Such a new particles is the dark photon. Dark photons are hypothetical hidden sector (a collection of yet unobserved quantum fiels) particle. Interaction between hidden sector and SM are weak, indirect and would be mediated via gravity or new particles. Dark photons are proposed as a force carrier, similar to the photon in the SM with a new abelian U(1) gauge symmetry. This new spin-1 gauge boson could couple very weakly to electrically charged patricles through kinetic mixing with the normal photon. \cite{noauthor_dark_2019}

%********************************** % Nomenclature  *************************************
\nomenclature[z-PBC]{PBC}{Physics Beyond Colliders}
\nomenclature[z-CERN]{CERN}{Conseil Européen pour la Recherche Nucléaire}
\nomenclature[z-LHC]{LHC}{Large Hadron Collider}
\nomenclature[z-HL]{HL}{High Luminosity}
\nomenclature[z-ATLAS]{ATLAS}{A Toroidal LHC ApparatuS}
\nomenclature[z-eV]{eV}{Electronvolt}
\nomenclature[a-pt]{$p_{T}$}{Transverse Momentum}
\nomenclature[z-SM]{SM}{Standard Model}
\nomenclature[z-IP]{IP}{Interaction Point}